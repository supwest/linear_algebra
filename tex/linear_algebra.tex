% \documentclass[xcolor=dvipsnames]{beamer}
% \usepackage{beamerthemesplit}
% \usepackage{bm,amsmath,color}
% \definecolor{darkblue}{rgb}{0.0,0.0,0.50}
% \definecolor{myorange}{cmyk}{0,0.7,1,0}
% \hypersetup{colorlinks = true, linkcolor=darkblue, citecolor=darkblue, 
% urlcolor=darkblue}
% \hypersetup{pdfauthor={Samwise the great}, pdftitle={Really it is a tale of samwise}}
% 
% %% create a color to highlight things
% \newcommand{\highlt}{\textcolor {myorange}}
% 
% %% set beamer theme and color change them as you want
% \usetheme{Frankfurt}
% \usecolortheme{rose}
% \setbeamertemplate{blocks}[rounded][shadow=true]

%% beamer packages
% other themes: AnnArbor, Antibes, Bergen, Berkeley, Berlin, Boadilla, boxes, 
% CambridgeUS, Darmstadt, Dresden, Frankfurt, Goettingen, Hannover, Ilmenau,
%JuanLesPins, Luebeck, Madrid, Malmoe, Marburg, Montpellier, PaloAlto,
%Pittsburgh, Rochester, Singapore, Szeged, Warsaw
% other colors: albatross, beaver, crane, default, dolphin, dove, fly, lily, 
%orchid, rose, seagull, seahorse, sidebartab, structure, whale, wolverine,
%beetle

\documentclass[xcolor=dvipsnames]{beamer}
\usepackage{beamerthemesplit}
\usepackage{bm,amsmath,marvosym}
\usepackage{listings,color}%xcolor
\usepackage[utf8]{inputenc}
\definecolor{shadecolor}{rgb}{.9, .9, .9}
\definecolor{darkblue}{rgb}{0.0,0.0,0.5}
\definecolor{myorange}{cmyk}{0,0.7,1,0}
%\definecolor{mypurple}{cmyk}{0.5,1.0,0,0.2}
\definecolor{mypurple}{cmyk}{0.3, 0.9, 0.0, 0.2}

\lstnewenvironment{code}{
    \lstset{backgroundcolor=\color{shadecolor},
        showstringspaces=false,
        language=python,
        frame=single,
        framerule=0pt,
        keepspaces=true,
        breaklines=true,
        basicstyle=\ttfamily,
        keywordstyle=\bfseries,
        basicstyle=\ttfamily\scriptsize,
        keywordstyle=\color{blue}\ttfamily,
        stringstyle=\color{red}\ttfamily,
        commentstyle=\color{green}\ttfamily,
        columns=fullflexible
    }
}{}

\lstnewenvironment{codeout}{
    \lstset{backgroundcolor=\color{shadecolor},
        frame=single,
        framerule=0pt,
        breaklines=true,
        basicstyle=\ttfamily\scriptsize,
        columns=fullflexible
    }
}{}

\hypersetup{colorlinks = true, linkcolor=darkblue, citecolor=darkblue,urlcolor=darkblue}
\hypersetup{pdfauthor={C. West}, pdftitle={linear algebra and numpy}}

\newcommand{\rd}{\textcolor{red}}
\newcommand{\grn}{\textcolor{green}}
\newcommand{\keywd}{\textcolor{myorange}}
\newcommand{\highlt}{\textcolor{darkblue}}
\newcommand{\norm}[1]{\left\lVert#1\right\rVert}
\def\ci{\perp\!\!\!\perp}

% set beamer theme and color
\usetheme{Frankfurt}
%\usetheme{Boadilla}
%\usecolortheme{dolphin}
\usecolortheme{seagull}
%\setbeamertemplate{blocks}[rounded][shadow=true]

\title[Linear Algebra and Numpy]{Linear Algebra and Numpy}
\author[]{}
%\institute{The Prancing Pony}
\date[\today]{Last updated: \today}

%%%%%%%%%%%%%%%%%%%%%%%%%%%%%%%%%%%%%%%%%%%%%%%%%%%%%%%%%%%%%%%%%%%%%%%%%%%%%%%
\begin{document}
\frame{\titlepage}
%%%%%%%%%%%%%%%%%%%%%%%%%%%%%%%%%%%%%%%%%%%%%%%%%%%%%%%%%%%%%%%%%%%%%%%%%%%%%%%
\frame{
\footnotesize
\tableofcontents
\normalsize
}

%%%%%%%%%%%%%%%%%%%%%%%%%%%%%%%%%%%%%%%%%%%%%%%%%%%%%%%%%%%%%%%%%%%%%%%%%%%%%%%
\section{Vectors}
\subsection{Linear Algebra}
%%%%%%%%%%%%%%%%%%%%%%%%%%%%%%%%%%%%%%%%%%%%%%%%%%%%%%%%%%%%%%%%%%%%%%%%%%%%%%%

\frame{
    \frametitle{Motivation for Linear Alegebra}
    \begin{itemize}
        \item Linear algebra is the mathematics of vectors
        \item It is used in many aspects of Machine Learning
            \begin{itemize}
                \item Dimensionality Reduction
                \item Recommender Systems
                \item Classification
                \item Clustering
            \end{itemize}
        \item So it's important to be familiar with the concept
    \end{itemize}
}


%\frame{   
%\frametitle{Motivation for Linear Algebra}
%\begin{itemize}
%\item <1-> a simple Markov Chain Monte Carlo (MCMC) algorithm 
%  	\begin{enumerate}
%  	 	\item choose some initial value
%  		\item do something with it
%	\end{enumerate}
%  \item <2-> another something else
%
%\vspace{2cm}
%  
%\begin{flushleft}
% \tiny{this only works in ``presentation'' mode}
%\end{flushleft}  
%\end{itemize}
%}

%%%%%%%%%%%%%%%%%%%%%%%%%%%%%%%%%%%%%%%%%%%%%%%%%%%%%%%%%%%%%%%%%%%%%%%%%%%%%%%
\frame{
\frametitle{Vectors}
A vector can be reprsented by an array of real numbers:
\begin{equation*}
    \mathbf{x}=\left[x_1, x_2, \dots, x_n\right]
\end{equation*}

A vector specifies a point in a (vector) space.

Can think of a vector as an arrow with its tail at the origin

The vector the gives the location of the tip of the arrow

The number of elements in the vector define the dimension of the vector
\begin{itemize}
    \item $\left[2, 2\right]$ is a two dimensional vector
    \item $\left[x_1, x_2, \dots, x_n\right]$ is an n-dimensional vector
    \item $\left[4\right]$ is a one dimensional vector, called a scalar
\end{itemize}
}
%We can arrange things into colums... really anything so a little figure on the left and a description on the right or a table
%you may need to play with column size
%\vspace{.5cm}
%\begin{columns}
%\begin{column}{5cm}
%\begin{center}
%\underline{Sequences} \\
%15264 Forward \\
%15270 Reverse \\
%2344 Failed QC \\
%84570 Total ORFs \\
%32322 After trimming \\
%414124 After digestion
%\end{center}
%\end{column}
%\begin{column}{5cm}
%\begin {itemize}
%\item la la la
%\item la ti di da 
%\item another la la 
%\end{itemize}
%\end{column}
%\end{columns}


\frame{
    \frametitle{Vector Operations}
    Now that we know what a vector is, what can we do with them?
    \begin{itemize}
        \item <1-> We can add a scalar
            \begin{itemize}
                \item $\mathbf{x} + a = \left[x_{1} + a, x_{2} + a,  + \dots x_{n} + a\right]$
            \end{itemize}
        \item <2-> We can multiply by a scalar
            \begin{itemize}
                \item $a\mathbf{x}= \left[ax_{1}, ax_{2}, \dots, ax_{n}\right]$
            \end{itemize}
        \item <3-> We can add and subtract them
            \begin{itemize}
                \item $\mathbf{x} + \mathbf{y} = \left[x_{1}+y_{1}, x_{2}+y_{2}, \dots, x_{n}+y_{n}\right]$
                \item $\mathbf{x} - \mathbf{y} = \left[x_{1}-y_{1}, x_{2}-y_{2}, \dots, x_{n}-y_{n}\right]$
                \item Note: $\mathbf{x}$ and $\mathbf{y}$ must be the same dimension for this to work
                \item This is know as a linear combination of $\mathbf{x}$ and $\mathbf{y}$
            \end{itemize}

    \end{itemize}

}

\frame{
    \frametitle{Dot Product}
    The dot product of two vectors $\mathbf{x}$ and $\mathbf{y}$ is 
    \begin{equation*}
        \mathbf{x}\cdot\mathbf{y} = \sum_i{ x_{i}y_{i}}
    \end{equation*}

    Note that $\mathbf{x}$ and $\mathbf{y}$ must be the same dimension

    This operations takes two vectors and returns a scalar

    If the dot product of two vectors is zero then the vectors are \textit{orthogonal}, or perpendicular
}

\frame{
    \frametitle{Norm of a Vector}
    The norm of a vector $\mathbf{x}$ is defined as
    \begin{equation*}
        \lVert x \rVert = \sqrt{x_{1}^{2} + x_{2}^{2} + \dots x_{n}^{2}}
    \end{equation*}

    This gives us the length, or magnitude of the vector.

    This is called the $l_{2}$ norm, in general the $l_{p}$ norm is
    \begin{equation*}
        \lVert x \rVert_p = \left(\lvert x_{1}\rvert^{p} + \lvert x_{2}\rvert^{p} + \dots + \lvert x_{n}\rvert^{p}\right)^{\frac{1}{p}}
    \end{equation*}

    \begin{itemize}
        \item <1-> When does this come up in machine learning?
        \item <2-> Normalization
            \begin{itemize}
                \item Dividing a vector by its norm $\mathbf{\frac{x}{\lVert x\rVert}}$ makes it a \textit{unit vector} (i.e. $\lVert\mathbf{u}\rVert = 1$
                \item Puts all variables on scale from [0, 1]
            \end{itemize}
        \item <3-> Regularization
            \begin{itemize}
                \item Helps prevent overfitting
            \end{itemize}
    \end{itemize}
}

\frame{
    \frametitle{Using the Norm: Distance and Angles}
    \begin{itemize}
        \item The norm can help us calculate the difference between two vectors:
        \begin{equation*}
            d\left(\mathbf{x}, \mathbf{y}\right) = \lVert \mathbf{x}-\mathbf{y} \rVert
        \end{equation*}
        \item It can also help us calculate the angle $\theta$ beteen two vectors (along with the dot product)
        \begin{equation*}
            cos\left(\theta\right) = \mathbf{\frac{x \cdot y}{\lVert x \rVert \lVert y \rVert}}
        \end{equation*}
            \begin{itemize}
                \item This is called the \textit{cosine similarity}
                \item Ranges between [-1, 1]
            \end{itemize}
    \end{itemize}
    }
   

\frame{
    \frametitle{Similarity}
    \begin{itemize}
        \item <1-> We can use both distance and angle to measure similarity
        \item <1-> Two vectors are similar if the distance between them is ``small''
        \item <1-> Two vectors are similar if their cosine similarity is close to 1
        \item <1-> When would this be useful in machine learning?
            \begin{itemize}
                \item <2-> Classification
                \item <2-> Clustering
                \item <2-> Recommender systems
            \end{itemize}
    \end{itemize}
}

\frame{
    \frametitle{Linear Combination}
    \begin{itemize}
        \item For some vectors $\left(\mathbf{x}_1, \mathbf{x}_2, \dots \mathbf{x}_k\right)$ if one vector $\mathbf{x_i}$ can be written as
            \begin{equation*} 
                \mathbf{x}_i = a_{1}\mathbf{x}_1 + a_2\mathbf{x}_2 + \dots +a_{i-1}\mathbf{x}_{i-1} + a_{i+1}\mathbf{x}_{i+i} + \dots + a_k\mathbf{x}_k
            \end{equation*}

            Then $\mathbf{x}_i$ is a linear combination of the vectors $\left(\mathbf{x}_1, \mathbf{x}_2, \dots \mathbf{x}_{i-1} , \mathbf{x}_{i+1}, \dots , \mathbf{x}_k\right)$


        \item We say that the vectors are \textit{linearly dependent}
    \end{itemize}
}
%%%%%%%%%%%%%%%%%%%%%%%%%%%%%%%%%%%%%%%%%%%%%%%%%%%%%%%%%%%%%%%%%%%%%%%%%%%%%%%
\section{Matrices}
\subsection{}
%%%%%%%%%%%%%%%%%%%%%%%%%%%%%%%%%%%%%%%%%%%%%%%%%%%%%%%%%%%%%%%%%%%%%%%%%%%%%%%%
\frame{
    \frametitle{Definition}
    A matrix is an array of numbers with $n$ rows and $p$ columns:
        \begin{equation*}
            X = \left[
            \begin{matrix} 
                x_{11} & x_{12} & \dots & x_{1p }\\
                x_{21} & x_{22} & \dots & x_{2p} \\
                \vdots & \vdots & \ddots & \vdots \\
                x_{n1} & x_{n2} & \dots & x_{np}
            \end{matrix}
            \right]
        \end{equation*}

    The dimension of $X$ is $n \times p$

    $x_{ij}$ refers to the element in the $i^{th}$ row and $j^{th}$ column
}

\frame{
    \frametitle{Basic Properties}
    If $X$ and $Y$ are both $n \times p$ matrices then
    \begin{itemize}
        \item $X+Y$ is a matrix whose $\left(i, j\right)^{th}$ entry is $x_{ij}+y_{ij}$
        \item $X-Y$ is a matrix whose $\left(i, j\right)^{th}$ entry is $x_{ij}-y_{ij}$
        \item $aX$ is a matrix whose $\left(i, j\right)^{th}$ entry is $ax_{ij}$
    \end{itemize}
}

\frame{
    \frametitle{Matrix Multiplication}
    \begin{itemize}
        \item In order to multiply two matrices they must be \textit{conformable}


        \item This means the number of columns of the first matrix must be the same as the number of rows of the second


        \item If two matrices are conformable then the product of $XY$ is a matrix $M$ whose $\left(i, j\right)^{th}$ element is the dot product of the $i^{th}$ row of $X$ with the $j^{th}$ column of $Y$
    \begin{equation*}
        m_{ij} = \sum _{s=1} ^{k} x_{is}y_{sj} = x_{i1}y_{1j} + \dots + x_{ik}y_{kj}
    \end{equation*}
\item Note that even if both $XY$ and $YX$ exist, in general $XY \ne YX$
    \end{itemize}
}

\frame{
    \frametitle{Additonal Properties}
    \begin{itemize}
        \item If $X$ and $Y$ are both $n \times p$ matrices, then $X+Y = Y+X$
\item If $X$, $Y$, and $Z$ are all $n \times p$ matrices, then $X+\left(Y+Z\right) = \left(X+Y\right)+Z$
\item If $X$, $Y$, and $Z$ are all conformable, then $X\left(YZ\right) = \left(XY\right)Z$
\item If $X$ is of dimension $n \times k$ and $Y$ and $Z$ are of dimension $k \times p$, then $X\left(Y+Z\right) = XY + XZ$
\item If $X$ is of dimension $p \times n$ and $Y$ and $Z$ are of dimension $k \times p$, then $\left(Y+Z\right)X = YX + ZX$
\item If $a$ and $b$ are real numbers, and $X$ is an $n \times p$ matrix, then $\left(a+b\right)X = aX+bX$
\item If $a$ is a real number, and $X$ and $Y$ are both $n \times p$ matrices, then $a\left(X+Y\right) = aX+aY$
\item If $z$ is a real number, and $X$ and $Y$ are conformable, then $X\left(aY\right) = a\left(XY\right)$
    \end{itemize}
}

\frame{
    \frametitle{Transpose}
     If $X$ is an $n \times p$ matrix
     \begin{equation*}
            X = \left[
            \begin{matrix} 
                x_{11} & x_{12} & \dots & x_{1p }\\
                x_{21} & x_{22} & \dots & x_{2p} \\
                \vdots & \vdots & \ddots & \vdots \\
                x_{n1} & x_{n2} & \dots & x_{np}
            \end{matrix}
            \right]
        \end{equation*}
        Then the transpose of $X$ is a $p \times n$ matrix with the rows and columns of $X$ interchanged
    \begin{equation*}
        X^{T} = \left[
            \begin{matrix} 
                x_{11} & x_{21} & \dots & x_{n1}\\
                x_{12} & x_{22} & \dots & x_{n2} \\
                \vdots & \vdots & \ddots & \vdots \\
                x_{1p} & x_{2p} & \dots & x_{np}
            \end{matrix}
            \right]
        \end{equation*}
        }

\frame{
    \frametitle{Properites of Transpose}
    \begin{itemize}
        \item Let $X$ be an $n \times p$ matrix and $a$ a real number, then 
$$(cX)^T = cX^T$$
\item Let $X$ and $Y$ be $n \times p$ matrices, then
$$(X \pm Y)^T = X^T \pm Y^T$$
\item Let $X$ be an $n \times k$ matrix and $Y$ be a $k \times p$ matrix, then
$$(XY)^T = Y^TX^T$$
    \end{itemize}
}

\frame{
    \frametitle{Vectors in Matrix Form}
    It can be useful to think of vectors as matrices
    \begin{itemize}
        \item A column vector is an $n \times 1$ matrix
            \begin{equation*}
                \mathbf{x} = 
                \left[
                    \begin{matrix}
                        x_{1}\\
                        x_{2}\\
                        \vdots\\
                        x_{n}
                    \end{matrix}
                    \right]
            \end{equation*}
        \item A row vector is written as the transpose
            \begin{equation*}
                \mathbf{x}^{T} = 
                \left[
                    \begin{matrix}
                        x_{1} & x_{2} & \dots & x_{n}
                    \end{matrix}
                    \right]
            \end{equation*}
        \item If two vectors $\mathbf{x}$ and $\mathbf{y}$ have the same length then the dot product is matrix multiplication
            $$\mathbf{x}^{T}\mathbf{y} = x_{1}y_{1} + \dots + x_{n}y_{n}$$
    \end{itemize}
        }

\frame{
    \frametitle{Inverse of a Matrix}
    \begin{itemize}
        \item The inverse of an $n \times n$ matrix $X$ is a matrix $X^{-1}$ of the same dimension where
            $$X^{-1}X = XX^{-1} = I$$
        \item $I$ is the identity matrix
        \item if $X^{-1}$ exists then $X$ is said to be \textit{invertible} or \textit{nonsingular}, otherwise $X$ is \textit{noninvertible} or \textit{singular}
        \item If any row (column) of $X$ can be represented as a linear combination of the other rows (columns) of $X$, then $X$ is singular
            \begin{itemize}
                \item The number of linearly independent columns or rows of $X$ is called the rank of $X$
                \item If $rank\left(X\right) < n$ then $X$ is singular
                \item If $rank\left(X\right) = n$ the $X$ is nonsingular
            \end{itemize}
    \end{itemize}
}

\frame{
    \frametitle{Properties of the Inverse}
    \begin{itemize}
        \item If $X$ is invertible, then $X^{-1}$ is invertible and
$$(X^{-1})^{-1} = X$$
\item If $X$ and $Y$ are both $n \times n$ invertible matrices, then $XY$ is invertible and $$(XY)^{-1} = Y^{-1}X^{-1}$$
\item If $X$ is invertible, then $X^T$ is invertible and $$\left(X^T\right)^{-1} = \left(X^{-1}\right)^T$$
    \end{itemize}
    }

\frame{
    \frametitle{Matrix Equations}
A system of equations of the form:
\begin{align*}
    a_{11}x_1 + \cdots + a_{1n}x_n &= b_1 \\
    \vdots \hspace{1in} \vdots \\
    a_{m1}x_1 + \cdots + a_{mn}x_n &= b_m 
\end{align*}
can be written as a matrix equation:
$$
A\mathbf{x} = \mathbf{b}
$$
and hence, has solution
$$
\mathbf{x} = A^{-1}\mathbf{b}
$$
}

\frame{
    \frametitle{Eigenvectors and Eigenvalues}
    Let $A$ be an $n \times n$ matrix and $\mathbf{x}$ be an $n \times 1$ nonzero vector. An \textit{eigenvalue} of $A$ is a number $\lambda$ such that

$$A \mathbf{x} = \lambda \mathbf{x}$$


A vector $\mathbf{x}$ satisfying this equation is called an \textit{eigenvector} associated with $\lambda$

Eigenvectors and eigenvalues will play a huge role in matrix methods later in the course (PCA, SVD, NMF).
}
    
%\frametitle{Example table}
%\begin{table}
%	\begin{center}
%		\textbf{Descriptive Statistics}
%	\end{center}
%	\begin{tabular}{llll}
%	\hline
%	Stat & Cleaves & Length &Mass \\
%	\hline
%	Min & 1 & 3 & $500^*$ \\
%	Max & 51 & 192 &  20446 \\
%	Median & 10 & 12 & 1130  \\
%	\hline
%	\end{tabular}
%	\label{table1}
%\end{table}
%$^*$ \small 500 Dalton is the suggested minimum peptide mass
%}

%%%%%%%%%%%%%%%%%%%%%%%%%%%%%%%%%%%%%%%%%%%%%%%%%%%%%%%%%%%%%%%%%%%%%%%%%%%%%%%%
%\frame
%{
%  \frametitle{The block thing}
%  \begin{block}{I am a block}
%   you can put what ever you want in a block like bullets.  Here is an example citation. 
%  \end{block}
%
%  \begin{enumerate}
%  \item \highlt{something} - is defined as something
%  \item \keywd{anther something} - is defined as something else         
%  \item \highlt{that something} - and so on  
%  \end{enumerate}
%
%\vspace{3cm}
%\begin{flushleft}
%\tiny{Anoter book but not as good as the one Bilbo wrote \tiny{\cite{Mount04}}}
%\end{flushleft}
%}
%
%%%%%%%%%%%%%%%%%%%%%%%%%%%%%%%%%%%%%%%%%%%%%%%%%%%%%%%%%%%%%%%%%%%%%%%%%%%%%%%%
%\begin{frame}[fragile]
%\scriptsize
%\frametitle{How to use verbatim}
%\begin{verbatim}
%#!/usr/bin/perl -w
%# Some code or something  
%
%print "\n", "I can write anything here!!!", "\n";
%\end{verbatim}
% \begin{flushleft}
%  See \href{http://www.columbia.edu/~jwp2128/Teaching/E6892/papers/mlss2007.pdf}{a lecture on Dirichlet processes} and \cite{Knuth92} for more information
% \end{flushleft}
%\end{frame}
%
%
%%%%%%%%%%%%%%%%%%%%%%%%%%%%%%%%%%%%%%%%%%%%%%%%%%%%%%%%%%%%%%%%%%%%%%%%%%%%%%%%
%% \frame{
%%  \frametitle{Including plots}
%%  \scriptsize
%% 
%%  \begin{center}
%%  \includegraphics[scale=0.3]{Someplot.png}
%% \end{center}
%% 
%% \begin{flushleft}
%%  \tiny{\cite{Mount04}}
%% \end{flushleft}
%%  \normalsize
%% }
%
%%%%%%%%%%%%%%%%%%%%%%%%%%%%%%%%%%%%%%%%%%%%%%%%%%%%%%%%%%%%%%%%%%%%%%%%%%%%%%%%%%%%%%%%%%%%
%\section{Rohan}
%\subsection{}
%%%%%%%%%%%%%%%%%%%%%%%%%%%%%%%%%%%%%%%%%%%%%%%%%%%%%%%%%%%%%%%%%%%%%%%%%%%%%%%%
%\frame{
%\frametitle{Some math}
%\footnotesize
%This approach, inspired by Yan and Nielsen, uses a set of 61 codon fitness parameters.
%\begin{equation} 
% \psi = (\psi_{a}) \text{ where } a \in \{1 \ldots 61 \} \text{ and } \sum a = 1 
%\end{equation}
%
%\begin{equation}
% Q_{ab} = \left\{
% \begin{array}{l l}
% \rho_{a_{c} b_{c}} \varphi_{b_{c}} \left( \frac{\psi_{b}}{\psi_{a}} \right)^{1/2},                        
%     & \quad \text{\tiny{if a and b are synonymous and differ only at} } c^{\text{th}} \text{\tiny{codon position}}\\
% \omega \rho_{a_{c} b_{c}} \varphi_{b_{c}} \left( \frac{\psi_{b}}{\psi_{a}} \right)^{1/2}, 
%     & \quad \text{\tiny{if a and b are nonsynonymous and differ only at} } c^{\text{th}} \text{\tiny{codon position}}\\
%    0,                                         
%     & \quad \text{\tiny{otherwise.}} 
%  \end{array} \right.
% \end{equation}
%With or without $\omega$ the models are referred to as Codon Preference (CP).  Therefore we have MG-CP, MG-NS-CP, and MG-NSDP-CP.
%}
%
%%%%%%%%%%%%%%%%%%%%%%%%%%%%%%%%%%%%%%%%%%%%%%%%%%%%%%%%%%%%%%%%%%%%%%%%%%%%%%%
%\begin{frame}[fragile]
% \frametitle{some example code and output}
%\tiny
% \scriptsize
%\begin{code}
%from sklearn.grid_search import GridSearchCV
% 
%random_forest_grid = {'max_depth': [3, None],
%                      'max_features': ['sqrt', 'log2', None],
%                      'min_samples_split': [1, 2, 4],
%                      'min_samples_leaf': [1, 2, 4],
%                      'bootstrap': [True, False],
%                      'n_estimators': [20, 40, 60, 80, 100, 120],
%                      'random_state': [42]}
%
%rf_gridsearch = GridSearchCV(RandomForestClassifier(),
%                             random_forest_grid,
%                             n_jobs=-1,verbose=True,
%                             scoring='f1_weighted')
%rf_gridsearch.fit(X_train, y_train)
%print "best parameters:", rf_gridsearch.best_params_
% 
% 
%\end{code}
% 
%\begin{codeout}
% best parameters: {'bootstrap': True, 'min_samples_leaf': 1, 'n_estimators': 100, 'min_samples_split': 1, 'random_state': 42, 'max_features': 'sqrt', 'max_depth': None}
%\end{codeout}
%\end{frame}
%
%%%%%%%%%%%%%%%%%%%%%%%%%%%%%%%%%%%%%%%%%%%%%%%%%%%%%%%%%%%%%%%%%%%%%%%%%%%%%%%
%\frame{
%\scriptsize
%\begin{enumerate}
% \item Don't forget that itemize/enumerate can be nested
% \item For each classifier:
%     \begin{itemize}
%     \scriptsize
%      \item Fit the classifier to the training data by minimizing the weighted error function
%      \begin{equation}
%       J_{m} = \sum^{N}_{n=1} w_{n}^{(m)} I(y_{m}(\mathbf{x}_{n}) \neq t_{n}) 
%      \end{equation}
%      \item Then evaluate
%      \begin{equation}
%       \epsilon_{m} = \frac{\sum^{N}_{n=1}w_{n}^{(m)} I(y_{m}(\mathbf{x}_{n}) \neq t_{n})}{\sum^{N}_{n=1}w_{n}^{(m)}}
%      \end{equation}
%      \item to get 
%      \begin{equation}
%      \alpha_{m} = \ln \left\{ \frac{1-\epsilon_{m}}{\epsilon_{m}} \right\}
%     \end{equation}
%     
%     \item Update the weighting coefficients
%     \begin{equation}
%      w_{n}^{(m+1)} = w_{n}^{(m)} \textrm{exp} \{ \alpha_{m} I(y_{m}(\mathbf{x}_{n}) \neq t_{n}) \}
%     \end{equation}
%     
%     \end{itemize}
% \item Make predictions using the final model
% \begin{equation}
% Y_{M}(\mathbf{x}) = \textrm{sign} \left[ \sum^{M}_{m=1} \alpha_{m} C_{m} (x) \right]    
%\end{equation}
%\end{enumerate}
%}
%
%%%%%%%%%%%%%%%%%%%%%%%%%%%%%%%%%%%%%%%%%%%%%%%%%%%%%%%%%%%%%%%%%%%%%%%%%%%%%%%%%%%%%%%%%%%
%\section{References}
%%%%%%%%%%%%%%%%%%%%%%%%%%%%%%%%%%%%%%%%%%%%%%%%%%%%%%%%%%%%%%%%%%%%%%%%%%%%%%%%%%%%%%%%%%%
%
%\frame[allowframebreaks]{
%\frametitle{References}
%\tiny
%\begin{thebibliography}{9}
%\bibliographystyle{apalike}
%\setbeamertemplate{bibliography item}[article]
%\bibitem{Knuth92}  D.E. Knuth, \emph{Two notes on notation}, Amer. Math. Monthly \textbf{99} (1992), 403--422.
%\setbeamertemplate{bibliography item}[book]
%\bibitem{Mount04} D W Mount, \emph{Bioinformatics: Sequence and Genome Informatics}, 2nd Edition, 2004 Cold Spring Harbor Laboratory Press, Cold Spring Harbor, NY pp 549 - 611
%\setbeamertemplate{bibliography item}[online]
%\bibitem{BioPerl} BioPerl - \href{http://www.bioperl.org/wiki/Main\_Page}{http://www.bioperl.org/wiki/Main\_Page}
%\setbeamertemplate{bibliography item}[article]
%\bibitem{Mangalam02} M H Mangalam, The Bio* toolkits--a brief overview., Brief Bioinform. 2002 Sep;3(3):296-302.
%\bibitem{Stajich02} J E Stajich \textit{et al.},The Bioperl toolkit: Perl modules for the life sciences., Genome Res. 2002 Oct;12(10):1611-8.
%\end{thebibliography}
%\normalsize
%}
%
%%%%%%%%%%%%%%%%%%%%%%%%%%%%%%%%%%%%%%%%%%%%%%%%%%%%%%%%%%%%%%%%%%%%%%%%%%%%%%%%
%%\frame[allowframebreaks]{  
%%\begin{tiny} \bibliography{galvanize-refs.bib}
%%\bibliographystyle{apalike}          % Style BST file
%%\end{tiny}
%%}

\end{document}
