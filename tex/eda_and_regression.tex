% \documentclass[xcolor=dvipsnames]{beamer}
% \usepackage{beamerthemesplit}
% \usepackage{bm,amsmath,color}
% \definecolor{darkblue}{rgb}{0.0,0.0,0.50}
% \definecolor{myorange}{cmyk}{0,0.7,1,0}
% \hypersetup{colorlinks = true, linkcolor=darkblue, citecolor=darkblue, 
% urlcolor=darkblue}
% \hypersetup{pdfauthor={Samwise the great}, pdftitle={Really it is a tale of samwise}}
% 
% %% create a color to highlight things
% \newcommand{\highlt}{\textcolor {myorange}}
% 
% %% set beamer theme and color change them as you want
% \usetheme{Frankfurt}
% \usecolortheme{rose}
% \setbeamertemplate{blocks}[rounded][shadow=true]

%% beamer packages
% other themes: AnnArbor, Antibes, Bergen, Berkeley, Berlin, Boadilla, boxes, 
% CambridgeUS, Darmstadt, Dresden, Frankfurt, Goettingen, Hannover, Ilmenau,
%JuanLesPins, Luebeck, Madrid, Malmoe, Marburg, Montpellier, PaloAlto,
%Pittsburgh, Rochester, Singapore, Szeged, Warsaw
% other colors: albatross, beaver, crane, default, dolphin, dove, fly, lily, 
%orchid, rose, seagull, seahorse, sidebartab, structure, whale, wolverine,
%beetle

\documentclass[xcolor=dvipsnames]{beamer}
\usepackage{beamerthemesplit}
\usepackage{bm,amsmath,marvosym}
\usepackage{listings,color}%xcolor
\usepackage[utf8]{inputenc}
\definecolor{shadecolor}{rgb}{.9, .9, .9}
\definecolor{darkblue}{rgb}{0.0,0.0,0.5}
\definecolor{myorange}{cmyk}{0,0.7,1,0}
%\definecolor{mypurple}{cmyk}{0.5,1.0,0,0.2}
\definecolor{mypurple}{cmyk}{0.3, 0.9, 0.0, 0.2}

\lstnewenvironment{code}{
    \lstset{backgroundcolor=\color{shadecolor},
        showstringspaces=false,
        language=python,
        frame=single,
        framerule=0pt,
        keepspaces=true,
        breaklines=true,
        basicstyle=\ttfamily,
        keywordstyle=\bfseries,
        basicstyle=\ttfamily\scriptsize,
        keywordstyle=\color{blue}\ttfamily,
        stringstyle=\color{red}\ttfamily,
        commentstyle=\color{green}\ttfamily,
        columns=fullflexible
    }
}{}

\lstnewenvironment{codeout}{
    \lstset{backgroundcolor=\color{shadecolor},
        frame=single,
        framerule=0pt,
        breaklines=true,
        basicstyle=\ttfamily\scriptsize,
        columns=fullflexible
    }
}{}

\hypersetup{colorlinks = true, linkcolor=darkblue, citecolor=darkblue,urlcolor=darkblue}
\hypersetup{pdfauthor={C. West}, pdftitle={linear algebra and numpy}}

\newcommand{\rd}{\textcolor{red}}
\newcommand{\grn}{\textcolor{green}}
\newcommand{\keywd}{\textcolor{myorange}}
\newcommand{\highlt}{\textcolor{darkblue}}
\newcommand{\norm}[1]{\left\lVert#1\right\rVert}
\def\ci{\perp\!\!\!\perp}

% set beamer theme and color
\usetheme{Frankfurt}
%\usetheme{Boadilla}
%\usecolortheme{dolphin}
\usecolortheme{seagull}
%\setbeamertemplate{blocks}[rounded][shadow=true]

\title[EDA and Linear Regression]{EDA and Linear Regression}
\author[]{}
%\institute{The Prancing Pony}
\date[\today]{Last updated: \today}

%%%%%%%%%%%%%%%%%%%%%%%%%%%%%%%%%%%%%%%%%%%%%%%%%%%%%%%%%%%%%%%%%%%%%%%%%%%%%%%
\begin{document}
\frame{\titlepage}
%%%%%%%%%%%%%%%%%%%%%%%%%%%%%%%%%%%%%%%%%%%%%%%%%%%%%%%%%%%%%%%%%%%%%%%%%%%%%%%
\frame{
\footnotesize
\tableofcontents
\normalsize
}

%%%%%%%%%%%%%%%%%%%%%%%%%%%%%%%%%%%%%%%%%%%%%%%%%%%%%%%%%%%%%%%%%%%%%%%%%%%%%%%
\section{Overview}
\subsection{}
%%%%%%%%%%%%%%%%%%%%%%%%%%%%%%%%%%%%%%%%%%%%%%%%%%%%%%%%%%%%%%%%%%%%%%%%%%%%%%%

\frame{
    \frametitle{Overview}
    \begin{itemize}
        \item Exploratory Data Analysis
        \item Simple Linear Regression
        \item Multiple Linear Regression
        \item Assessing Fit
        \item Comparing Model
        \item Interpretation of Model Output
    \end{itemize}
}


%\frame{   
%\frametitle{Motivation for Linear Algebra}
%\begin{itemize}
%\item <1-> a simple Markov Chain Monte Carlo (MCMC) algorithm 
%  	\begin{enumerate}
%  	 	\item choose some initial value
%  		\item do something with it
%	\end{enumerate}
%  \item <2-> another something else
%
%\vspace{2cm}
%  
%\begin{flushleft}
% \tiny{this only works in ``presentation'' mode}
%\end{flushleft}  
%\end{itemize}
%}

\section{EDA}

%%%%%%%%%%%%%%%%%%%%%%%%%%%%%%%%%%%%%%%%%%%%%%%%%%%%%%%%%%%%%%%%%%%%%%%%%%%%%%%
\frame{
\frametitle{EDA}
    High level overview of a new dataset

    \begin{itemize}
        \item How are the data arranged
        \item What variables do we have: categorical vs. continuous
        \item Are there missing values
        \item What do the distributions look like
        \item How are features related
    \end{itemize}

}
%We can arrange things into colums... really anything so a little figure on the left and a description on the right or a table
%you may need to play with column size
%\vspace{.5cm}
%\begin{columns}
%\begin{column}{5cm}
%\begin{center}
%\underline{Sequences} \\
%15264 Forward \\
%15270 Reverse \\
%2344 Failed QC \\
%84570 Total ORFs \\
%32322 After trimming \\
%414124 After digestion
%\end{center}
%\end{column}
%\begin{column}{5cm}
%\begin {itemize}
%\item la la la
%\item la ti di da 
%\item another la la 
%\end{itemize}
%\end{column}
%\end{columns}


\frame{
    \frametitle{Data Cleaning/Munging}
    Most of the we'll have to clean the data we get
    \begin{itemize}
        \item <1-> Dirtiness - does the data make sense
           
        \item <2-> Missing Data
            \begin{itemize}
                \item <2-> What do we do?
                \item <3-> Drop rows with missing values
                \item <3-> Impute the missing values
            \end{itemize}
        \item <4-> Convert data types
        \item <5-> Transform data
    \end{itemize}

}

\frame{
    \frametitle{Types of Variables}
\begin{itemize}
    \item Qualitative (Categorical)
        \begin{itemize}
            \item Barcharts
        \end{itemize}
    \item Quantitative (Continuous)
        \begin{itemize}
            \item Histogram
            \item Scatterplot
            \item Boxplot
        \end{itemize}
\end{itemize}

We want to get an idea of what our variables look like


}

\section{Simple Linear Regression}

\frame{
    \frametitle{Simple Linear Regression}
    The idea is to describe a linear relationship between two variables

    \begin{itemize}
        \item Fuel milage and horsepower
        \item Income and savings
        \item On-base percentage and wins
        \item Etc.
    \end{itemize}

    We're going to do that by fitting a line to our data

}   
\frame{
    \frametitle{Linear Regression Model}
    The basic model is 
    $$Y = \beta_0 + \beta_{1}X + \varepsilon$$
    \begin{itemize}
        \item $\beta_0$ and $\beta_1$ are unknown constants that represent the intercept and slope of our line
        \item $\varepsilon$ is the error term
            \begin{itemize}
                \item $\varepsilon \sim i.i.d. N\left(0, \sigma^{2}\right)$
                \item This is the reason not all point are on the line
            \end{itemize}
        \item Since we don't know $\beta_0$ or $\beta_1$ we'll estimate them
        \item $\hat{y} = \hat{\beta}_0 + \hat{\beta}_{1}x$
            \begin{itemize}
                \item $\hat{\beta}_0, \hat{\beta}_1$ are our estimates
                \item $\hat{y}$ is our prediction
                    \end{itemize}
                \item Can think of $Y|_{X} \sim N\left(\beta_0 + \beta_{1}X, \sigma^2\right)$
    \end{itemize}


    }
   

\frame{
    \frametitle{Estimating Coefficients}
    We want to find the line that fits our data the ``best''

    If we define our residual as
    $$e_i = y_i - \hat{y}_i$$

    Then the best line is the one that minimizes the sum of the squared residuals
    \begin{equation*}
        RSS = \sum _i e_{}^{2} = \sum _i \left(y_i - \hat{\beta}_0 - \hat{\beta}_{1}x_i\right)^2
    \end{equation*}

    Sovling this equation gives us
    \begin{equation*}
        \hat{\beta}_0 = \overline{y} - \hat{\beta}_{1}\overline{x}
    \end{equation*}
    \begin{equation*}
        \hat{\beta}_1 = \frac{\sum \left(x_i - \overline{x}\right)\left(y_i - \overline{y}\right)}{\sum _i \left(x_i - \overline{x}\right)^2}
    \end{equation*}

    
}

\frame{
    \frametitle{Four Assumptions for Linear Regression}
    There are four assumptions that need to be met in order for our linear regression model to be valid
    \begin{itemize}
        \item <1-> Linearity
            \begin{itemize}
                \item <1-> The relationship between X and Y is linear
            \end{itemize}
        \item <2-> Constant Standard Deviation
            \begin{itemize}
                \item <2-> The standard deviation of $y$ does not depend on $X$
            \end{itemize}
        \item <3-> Independence
            \begin{itemize}
                \item <3-> The residuals are independent of X
            \end{itemize}
        \item <4-> Normality
            \begin{itemize} 
                \item <4-> The residuals are normally distributed
            \end{itemize}
    \end{itemize}
}

%%%%%%%%%%%%%%%%%%%%%%%%%%%%%%%%%%%%%%%%%%%%%%%%%%%%%%%%%%%%%%%%%%%%%%%%%%%%%%%
%\section{Matrices}
%\subsection{}
%%%%%%%%%%%%%%%%%%%%%%%%%%%%%%%%%%%%%%%%%%%%%%%%%%%%%%%%%%%%%%%%%%%%%%%%%%%%%%%%
\frame{
    \frametitle{Assessing Model Fit}
    Once we estimate a model we can judge how well it fits our data

    \begin{itemize}
        \item Look at statistical significance of our coefficients
            \begin{itemize}
                \item $H_{0}: \beta_i=0$
                \item Get p-value and CI for $\hat{\beta}_i$
            \end{itemize}
        \item Look at the significance of the model
            \begin{itemize}
                \item $H_{0}: \beta_1 = \beta_2 = \dots = \beta_{k} = 0$
                \item This is done with an F-test
            \end{itemize}
        \item Look at fit statistics
    \end{itemize}
}

\frame{
    \frametitle{Significance of Coefficients}
        For each of our coefficent estimates we can perform a hypothesis test
        \begin{itemize}
            \item $H_0: \beta_{1} = 0$
            \item Test statistic is $$\frac{\hat{\beta}_{1} - 0}{SE\left(\hat{\beta}_1\right)}$$
            \item CI is $$\hat{\beta_{1}} \pm t_{\frac{\alpha}{2}}*SE\left(\hat{\beta}_{1}\right)$$
        \end{itemize}

        If p-value is less than $\alpha$ then coefficient is statitically significant
        \begin{itemize}
            \item The associated X variable has some explanatory power
        \end{itemize}
}

\frame{
    \frametitle{Significance of Regression}
    For multiple regression we can test the signficance of the regression as a whole
    \begin{itemize}
        \item Is it even worth doing a regression analysis at all
    \end{itemize}

    We do this with a F-test

    \begin{itemize}
        \item $H_{0}: \beta_1 = \beta_{2} = \dots = \beta_{k}$
        \item Test statistic is $$ F = \frac{\left(TSS - RSS\right)/k}{RSS/\left(n-k-1\right)} \sim F_{k, n-k-1}$$
        \item TSS is the Total Sum of Squares = $\sum \left(y_i - \overline{y}_i\right)$
        \item If we reject this null, then at least one of our $X$ variables has some explanatory power
    \end{itemize}
    
    The F-test can also be used to test the significance of a subset of our X variables
    
}

\frame{
    \frametitle{Fit Statistics}
    \begin{itemize}
        \item RSS is the Residual Sum of Squares
            \begin{itemize}
                \item The variation in $y$ that is unexplained by $X$
                \item Not very informative (increases with n)
            \end{itemize}
        \item MSE is $\frac{RSS}{n-k-1}$
            \begin{itemize}
                \item ``Average`` unexplained error
            \end{itemize}
        \item $R^2$ is $\frac{TSS-RSS}{TSS} = 1 - \frac{RSS}{TSS}$
            \begin{itemize}
                \item Proportion of variation in y explained by variation in X
            \end{itemize}
    \end{itemize}

    
}

\frame{
    \frametitle{Comparing Multiple Models}
    How do we decide which variables to include in our model?
   
    We could pick the model with the highest $R^{2}$
    \begin{itemize}
        \item <1-> Turns out not to be such a great idea
        \item <1-> Why?
        \item <2-> $R^{2}$ will never decrease
    \end{itemize}

    One solution is to look at the Adjusted $R^2$
    \begin{itemize}
        \item $Adj. R^{2} = 1 - \left(1 - R^{2}\right)\frac{n-1}{n-k-1}$
        \item Penalizes $R^{2}$ for including extra variables
    \end{itemize}

    There are other ways as well
}
\frame{
    \frametitle{F-test}
    Suppose we have a model for gas milage
    \begin{equation*}
        Y_{full} = \beta_0 + \beta_{1}weight + \beta_{2}horsepower + \beta_{3}color + \beta_{4}height
    \end{equation*}
    But we suspect height and color might not be important, so we can consider
    \begin{equation*}
        Y_{reduced} = \beta_{0} + \beta_{1}weight + \beta_{2}horsepower
    \end{equation*}

    we can use an F-test to test $H_0:\beta_{3} = \beta_{4} = 0$
    \begin{equation*}
        F = \frac{\left(RSS_{reduced} - RSS_{full}\right)/\left(k_{full}-k_{reduced}\right)}{RSS_{full}/\left(n-k_{full}-1\right)}
    \end{equation*}
    The idea is that if $\beta_{3}$ and $\beta_{4}$ don't matter, then $\left(RSS_{reduced} - RSS_{full}\right)$ will be small, so $F$ will be small

}

\frame{
    \frametitle{AIC and BIC}
    Additionally we can look at the AIC and BIC for the model
    \begin{itemize}
        \item Akaike Information Criterion = $2k - 2 ln\left(\mathcal{L}\right)$
        \item Bayesian Information Criterion = $-2 ln\left(\mathcal{L}\right) + k ln\left(n\right)$
        \item $\mathcal{L}$ is the maximized value of the likelihood function
    \end{itemize}
    Both of these scores penalize models with more explanatory variables

    \begin{itemize}
        \item Question: Do we want lower or higher values of AIC/BIC?
    \end{itemize}
        }

\frame{
    \frametitle{Interpretation}
    Let's interpret some results
}
   
\section{Summary}

\frame{
    \frametitle{EDA Summary}
    EDA is a first look at the data
    \begin{itemize}
        \item Look at first few rows
        \item Plot variables to examine distributions/relationships
        \item What to do with missing data
        \item What else?
    \end{itemize}
}
\frame{
    \frametitle{Linear Regression Summary}

    Steps in Linear Regression
    \begin{itemize}
        \item Fit model
        \item Examine Residuals
        \item Examine Results
            \begin{itemize}
                \item Are all variables significant and make sense?
                \item If not, try other models
            \end{itemize}
        \item Examine Residuals
        \item Interpret Results
    \end{itemize}
}

%\frametitle{Example table}
%\begin{table}
%	\begin{center}
%		\textbf{Descriptive Statistics}
%	\end{center}
%	\begin{tabular}{llll}
%	\hline
%	Stat & Cleaves & Length &Mass \\
%	\hline
%	Min & 1 & 3 & $500^*$ \\
%	Max & 51 & 192 &  20446 \\
%	Median & 10 & 12 & 1130  \\
%	\hline
%	\end{tabular}
%	\label{table1}
%\end{table}
%$^*$ \small 500 Dalton is the suggested minimum peptide mass
%}

%%%%%%%%%%%%%%%%%%%%%%%%%%%%%%%%%%%%%%%%%%%%%%%%%%%%%%%%%%%%%%%%%%%%%%%%%%%%%%%%
%\frame
%{
%  \frametitle{The block thing}
%  \begin{block}{I am a block}
%   you can put what ever you want in a block like bullets.  Here is an example citation. 
%  \end{block}
%
%  \begin{enumerate}
%  \item \highlt{something} - is defined as something
%  \item \keywd{anther something} - is defined as something else         
%  \item \highlt{that something} - and so on  
%  \end{enumerate}
%
%\vspace{3cm}
%\begin{flushleft}
%\tiny{Anoter book but not as good as the one Bilbo wrote \tiny{\cite{Mount04}}}
%\end{flushleft}
%}
%
%%%%%%%%%%%%%%%%%%%%%%%%%%%%%%%%%%%%%%%%%%%%%%%%%%%%%%%%%%%%%%%%%%%%%%%%%%%%%%%%
%\begin{frame}[fragile]
%\scriptsize
%\frametitle{How to use verbatim}
%\begin{verbatim}
%#!/usr/bin/perl -w
%# Some code or something  
%
%print "\n", "I can write anything here!!!", "\n";
%\end{verbatim}
% \begin{flushleft}
%  See \href{http://www.columbia.edu/~jwp2128/Teaching/E6892/papers/mlss2007.pdf}{a lecture on Dirichlet processes} and \cite{Knuth92} for more information
% \end{flushleft}
%\end{frame}
%
%
%%%%%%%%%%%%%%%%%%%%%%%%%%%%%%%%%%%%%%%%%%%%%%%%%%%%%%%%%%%%%%%%%%%%%%%%%%%%%%%%
%% \frame{
%%  \frametitle{Including plots}
%%  \scriptsize
%% 
%%  \begin{center}
%%  \includegraphics[scale=0.3]{Someplot.png}
%% \end{center}
%% 
%% \begin{flushleft}
%%  \tiny{\cite{Mount04}}
%% \end{flushleft}
%%  \normalsize
%% }
%
%%%%%%%%%%%%%%%%%%%%%%%%%%%%%%%%%%%%%%%%%%%%%%%%%%%%%%%%%%%%%%%%%%%%%%%%%%%%%%%%%%%%%%%%%%%%
%\section{Rohan}
%\subsection{}
%%%%%%%%%%%%%%%%%%%%%%%%%%%%%%%%%%%%%%%%%%%%%%%%%%%%%%%%%%%%%%%%%%%%%%%%%%%%%%%%
%\frame{
%\frametitle{Some math}
%\footnotesize
%This approach, inspired by Yan and Nielsen, uses a set of 61 codon fitness parameters.
%\begin{equation} 
% \psi = (\psi_{a}) \text{ where } a \in \{1 \ldots 61 \} \text{ and } \sum a = 1 
%\end{equation}
%
%\begin{equation}
% Q_{ab} = \left\{
% \begin{array}{l l}
% \rho_{a_{c} b_{c}} \varphi_{b_{c}} \left( \frac{\psi_{b}}{\psi_{a}} \right)^{1/2},                        
%     & \quad \text{\tiny{if a and b are synonymous and differ only at} } c^{\text{th}} \text{\tiny{codon position}}\\
% \omega \rho_{a_{c} b_{c}} \varphi_{b_{c}} \left( \frac{\psi_{b}}{\psi_{a}} \right)^{1/2}, 
%     & \quad \text{\tiny{if a and b are nonsynonymous and differ only at} } c^{\text{th}} \text{\tiny{codon position}}\\
%    0,                                         
%     & \quad \text{\tiny{otherwise.}} 
%  \end{array} \right.
% \end{equation}
%With or without $\omega$ the models are referred to as Codon Preference (CP).  Therefore we have MG-CP, MG-NS-CP, and MG-NSDP-CP.
%}
%
%%%%%%%%%%%%%%%%%%%%%%%%%%%%%%%%%%%%%%%%%%%%%%%%%%%%%%%%%%%%%%%%%%%%%%%%%%%%%%%
%\begin{frame}[fragile]
% \frametitle{some example code and output}
%\tiny
% \scriptsize
%\begin{code}
%from sklearn.grid_search import GridSearchCV
% 
%random_forest_grid = {'max_depth': [3, None],
%                      'max_features': ['sqrt', 'log2', None],
%                      'min_samples_split': [1, 2, 4],
%                      'min_samples_leaf': [1, 2, 4],
%                      'bootstrap': [True, False],
%                      'n_estimators': [20, 40, 60, 80, 100, 120],
%                      'random_state': [42]}
%
%rf_gridsearch = GridSearchCV(RandomForestClassifier(),
%                             random_forest_grid,
%                             n_jobs=-1,verbose=True,
%                             scoring='f1_weighted')
%rf_gridsearch.fit(X_train, y_train)
%print "best parameters:", rf_gridsearch.best_params_
% 
% 
%\end{code}
% 
%\begin{codeout}
% best parameters: {'bootstrap': True, 'min_samples_leaf': 1, 'n_estimators': 100, 'min_samples_split': 1, 'random_state': 42, 'max_features': 'sqrt', 'max_depth': None}
%\end{codeout}
%\end{frame}
%
%%%%%%%%%%%%%%%%%%%%%%%%%%%%%%%%%%%%%%%%%%%%%%%%%%%%%%%%%%%%%%%%%%%%%%%%%%%%%%%
%\frame{
%\scriptsize
%\begin{enumerate}
% \item Don't forget that itemize/enumerate can be nested
% \item For each classifier:
%     \begin{itemize}
%     \scriptsize
%      \item Fit the classifier to the training data by minimizing the weighted error function
%      \begin{equation}
%       J_{m} = \sum^{N}_{n=1} w_{n}^{(m)} I(y_{m}(\mathbf{x}_{n}) \neq t_{n}) 
%      \end{equation}
%      \item Then evaluate
%      \begin{equation}
%       \epsilon_{m} = \frac{\sum^{N}_{n=1}w_{n}^{(m)} I(y_{m}(\mathbf{x}_{n}) \neq t_{n})}{\sum^{N}_{n=1}w_{n}^{(m)}}
%      \end{equation}
%      \item to get 
%      \begin{equation}
%      \alpha_{m} = \ln \left\{ \frac{1-\epsilon_{m}}{\epsilon_{m}} \right\}
%     \end{equation}
%     
%     \item Update the weighting coefficients
%     \begin{equation}
%      w_{n}^{(m+1)} = w_{n}^{(m)} \textrm{exp} \{ \alpha_{m} I(y_{m}(\mathbf{x}_{n}) \neq t_{n}) \}
%     \end{equation}
%     
%     \end{itemize}
% \item Make predictions using the final model
% \begin{equation}
% Y_{M}(\mathbf{x}) = \textrm{sign} \left[ \sum^{M}_{m=1} \alpha_{m} C_{m} (x) \right]    
%\end{equation}
%\end{enumerate}
%}
%
%%%%%%%%%%%%%%%%%%%%%%%%%%%%%%%%%%%%%%%%%%%%%%%%%%%%%%%%%%%%%%%%%%%%%%%%%%%%%%%%%%%%%%%%%%%
%\section{References}
%%%%%%%%%%%%%%%%%%%%%%%%%%%%%%%%%%%%%%%%%%%%%%%%%%%%%%%%%%%%%%%%%%%%%%%%%%%%%%%%%%%%%%%%%%%
%
%\frame[allowframebreaks]{
%\frametitle{References}
%\tiny
%\begin{thebibliography}{9}
%\bibliographystyle{apalike}
%\setbeamertemplate{bibliography item}[article]
%\bibitem{Knuth92}  D.E. Knuth, \emph{Two notes on notation}, Amer. Math. Monthly \textbf{99} (1992), 403--422.
%\setbeamertemplate{bibliography item}[book]
%\bibitem{Mount04} D W Mount, \emph{Bioinformatics: Sequence and Genome Informatics}, 2nd Edition, 2004 Cold Spring Harbor Laboratory Press, Cold Spring Harbor, NY pp 549 - 611
%\setbeamertemplate{bibliography item}[online]
%\bibitem{BioPerl} BioPerl - \href{http://www.bioperl.org/wiki/Main\_Page}{http://www.bioperl.org/wiki/Main\_Page}
%\setbeamertemplate{bibliography item}[article]
%\bibitem{Mangalam02} M H Mangalam, The Bio* toolkits--a brief overview., Brief Bioinform. 2002 Sep;3(3):296-302.
%\bibitem{Stajich02} J E Stajich \textit{et al.},The Bioperl toolkit: Perl modules for the life sciences., Genome Res. 2002 Oct;12(10):1611-8.
%\end{thebibliography}
%\normalsize
%}
%
%%%%%%%%%%%%%%%%%%%%%%%%%%%%%%%%%%%%%%%%%%%%%%%%%%%%%%%%%%%%%%%%%%%%%%%%%%%%%%%%
%%\frame[allowframebreaks]{  
%%\begin{tiny} \bibliography{galvanize-refs.bib}
%%\bibliographystyle{apalike}          % Style BST file
%%\end{tiny}
%%}

\end{document}
